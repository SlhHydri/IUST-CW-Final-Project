\documentclass[titlepage, 12pt]{article}

\usepackage{graphicx}

\usepackage{hyperref}

\usepackage{url}
\usepackage{tikz}
\usepackage{caption}

\usepackage{listings}

\topmargin=-0.45in
\evensidemargin=0in
\oddsidemargin=0in
\textwidth=6.5in
\textheight=9.0in
\headsep=0.25in

\newcommand{\HWAuthorName}{M. Saleh Heydari}
\newcommand{\HWTitle}{Final Assignment: Integration of Tools and Practices}
\newcommand{\HWDate}{Bahman 6, 1402}
\newcommand{\HWClass}{Computer Workshop}

\usepackage{fancyhdr}
\pagestyle{fancy}
\lhead{\textbf{\HWAuthorName}}
\chead{\HWClass: \ Final Assignment}
\rhead{\HWDate}
\cfoot{Page \thepage}

\newcommand{\code}{\texttt}

\title{\textbf{\HWClass \\ \HWTitle}}

\author{\HWAuthorName}
\date{\HWDate}


\begin{document}
	
	\maketitle
	
	\tableofcontents
	\pagebreak
	
	\section{Git and GitHub}
	\subsection{Repository Initialization and Commits}
	Write about how you set up the repository for this assignment. Explain every step in detail.
	\\
	\\
	\textbf{
		\underline{Step 1.} Went to \code{https://github.com/SlhHydri?tab=repositories}, pressed on the green \textit{``New"} button on the upper right side of the page.
		\\
		\\
		\underline{Step 2.} Named the new repository \textit{``IUST-CW-Final-Project Public"} and pressed the green \textit{``Create repository"} button at the bottom tight side of the page.
		\\
		\\
		\underline{Step 3.} Cloned the new repository to my machine using the command:
		\begin{itemize}
			\item \code{git clone https://github.com/SlhHydri/IUST-CW-Final-Project.git}
		\end{itemize}
		\underline{Step 4.} Initialized the repository. A summarized, edited series of commands that were used are as follows:
		\begin{itemize}
			\item   \code{echo "\# IUST-CW-Final-Project" >> README.md}
			\item	\code{git add README.md}
			\item	\code{git commit -m "first commit"}
			\item	\code{git push}
		\end{itemize}
		\underline{Step 5.} Initilized the \code{main.tex} file in my editor, added it to the repo using:
		\begin{itemize}
			\item \code{git add main.tex}
			\item \code{git push}
		\end{itemize}
		How the \code{.github/workflows} folder came to be, is explained in the next section!}
	
	
	\subsection{GitHub Actions for \LaTeX Compilation}
	Provide a walkthrough of setting up GitHub Actions to automatically compile your \LaTeX 
	document and any challenges you encountered. \\
	\textbf{It was hassle-free task thanks to the provided \code{.github/workflows} folder!
		\\
		But here's a simple way to do that: You can start by making the directory \code{.github/workflows}, downloading the file \code{main.yml} from the provided repository and moving it there, and pushing the changes you have done as of now in your local machine to the GitHub server using the command \code{git push}.\\
		The real challenge was the Action failing to compile the \code{main.tex} file, due to an error in the \TeX document (usage of \# without the backslash before it), which was fixed in the commits that followed.}
	
	\section{Exploration Tasks}
	\subsection{Vim Advanced Features}
	Explore and document 3 advanced features of Vim that were not covered in class.
	\textbf{
		\begin{enumerate}
			\item For times you may have opened a file without adding the \code{sudo} behind your command, made some changes, and now cannot save them, you can use the following command: \\ \code{:w !sudo tee \%}  
			\item For times you may want to convert all tabs to spaces in a file you are editing, you can use the following commands: \\
			\code{:set expandtab \\
				:set tabstop=4 \\
				:set shiftwidth=4 \\
				:retab}
			\item For times you want to spell-check your file---similar to what happens at applications like \textit{MS Word}---so the wrong-spelled words are highlighted, you can use the following command: \\
			\code{:set spell}
		\end{enumerate}
	}
\end{document}